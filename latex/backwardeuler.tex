%  ____             _                           _   _____      _
% | __ )  __ _  ___| | ____      ____ _ _ __ __| | | ____|   _| | ___ _ __
% |  _ \ / _` |/ __| |/ /\ \ /\ / / _` | '__/ _` | |  _|| | | | |/ _ \ '__|
% | |_) | (_| | (__|   <  \ V  V / (_| | | | (_| | | |__| |_| | |  __/ |
% |____/ \__,_|\___|_|\_\  \_/\_/ \__,_|_|  \__,_| |_____\__,_|_|\___|_|
%

\subsection{Backward Euler} \label{sec:backwardEuler}
The idea of backward Euler is similar to the forward Euler. Actually, the only difference lies in how we approximate the time derivative of \(u\) which will give us the freedom of choosing any \(\Delta t\) and \(\Delta x\) we like.
\subsubsection{Derivation and error analysis}
We can also use Taylor polynomials of \(u(x_i,t_{j-1})\)to write an approximation of the time derivative of \(u\) around \(x_i,t_j\):
\[
    u_{i,j-1} = u_{i,j} - \Delta t\pdv{u_{i,j}}{t} + \frac{1}{2}\qty(\Delta t)^2\pdv[2]{u(x_i,\tilde{t})}{t}
    \implies \pdv{u_{i,j}}{t} = \frac{u_{i,j}-u_{i,j-1}}{\Delta t} + \frac{1}{2}\Delta t\pdv[2]{u(x_i,\tilde{t})}{t}
\]
where \(\tilde{t}\) and the second term are the same as defined in \vref{sec:ForwardEulerDerivation}. \Vref{andrederivert} can be reused as an approximation to the second derivative.

With these approximations, the diffusion equation can be written as
\begin{alignat*}{2}
    \frac{u_{i,j}-u_{i,j-1}}{\Delta t} + \frac{1}{2}\Delta t\pdv[2]{u(x_j,\tilde{t})}{t}
    &=&& \frac{u_{i+1,j}  - 2u_{i,j} + u_{i-1,j}}{\qty(\Delta x)^2} + \frac{1}{12}\qty(\Delta x)^2\pdv[4]{u(\tilde{x},t_j)}{x}  \\
    u_{i,j}-u_{i,j-1} &=&& \alpha(u_{i+1,j}  - 2u_{i,j} + u_{i-1,j})\\
    &&& + \Delta t\cdot \frac{1}{12}\qty(\Delta x)^2\pdv[4]{u(\tilde{x},t_j)}{x}  - \frac{1}{2}\qty(\Delta t)^2\pdv[2]{u(x_j,\tilde{t})}{t}\\
    -\alpha u_{i-1,j} + \qty(1+2\alpha)u_{i,j} - \alpha u_{i+1,j} &=&& u_{i,j-1} + \Delta t\cdot \frac{1}{12}\qty(\Delta x)^2\pdv[4]{u(\tilde{x},t_j)}{x}  - \frac{1}{2}\qty(\Delta t)^2\pdv[2]{u(x_j,\tilde{t})}{t}
\end{alignat*}
In this equation, only the \(u\) on the right hand side is known, while the \(u\)'s on the left side are the ones we are interested in. As it is not possible to find an explicit expression for the quantities of interest, this leads to an implicit scheme. Observe that the error is the same as for the Forward Euler scheme.

To solve the equation above, note that it must hold for all \(i\in\qty[1,n-1]\cap\mathbb{N}\), giving the following set of equations:
\[
    \begin{array}{ccccccc}
        -\alpha \overbrace{u_{0,j}}^0 &+& (1+2\alpha)u_{1,j} &-& \alpha u_{2,j} &=& u_{1,j-1}\\
        -\alpha u_{1,j} &+& (1+2\alpha)u_{2,j} &-& \alpha u_{3,j} &=& u_{2,j-1}\\
        -\alpha u_{2,j} &+& (1+2\alpha)u_{3,j} &-& \alpha u_{4,j} &=& u_{3,j-1}\\
        \vdots && \vdots && \vdots && \vdots\\
        -\alpha u_{n-3,j} &+& (1+2\alpha)u_{n-2,j} &-& \alpha u_{n-1,j} &=& u_{n-2,j-1}\\
        -\alpha u_{n-2,j} &+& (1+2\alpha)u_{n-1,j} &-& \alpha \underbrace{u_{n,j}}_1 &=& u_{n-1,j-1}\\
    \end{array}
\]
To conclude what we found and taking away the truncation errors, we have the numerical approximation of \(u(x_i,t_{j-1})\) to be
\begin{equation}\label{eq:backwardEuler}
u_{i,j-1} = -\alpha u_{i-1,j} + \qty(1+2\alpha)u_{i,j} - \alpha u_{i+1,j}
\end{equation}
This can then be written on a matrix form:
\[
    \begin{bmatrix}
        1 + 2\alpha & -\alpha & 0 & 0 & 0 & \dots & 0 & 0 & 0 & 0  \\
        -\alpha & 1+2\alpha & -\alpha & 0 & 0 & \dots & 0 & 0 & 0 & 0  \\
        0 & -\alpha & 1+2\alpha & -\alpha & 0 & \dots & 0 & 0 & 0 & 0 \\
        \vdots & \vdots &  \ddots & \ddots & \ddots & \vdots & \vdots & \vdots & \vdots & \vdots\\
        0 & 0 & 0 & 0 & 0 & \dots & -\alpha & 1+2\alpha & - \alpha & 0\\
        0 & 0 & 0 & 0 & 0 & 0 & \dots & -\alpha & 1+2\alpha & - \alpha\\
        0 & 0 & 0 & 0 & 0 & 0 & 0 & \dots & -\alpha & 1+2\alpha\\
    \end{bmatrix}
    \begin{bmatrix}
        u_{1,j}\\
        u_{2,j}\\
        u_{3,j}\\
        \vdots\\
        u_{n-3,j}\\
        u_{n-2,j}\\
        u_{n-1,j}\\
    \end{bmatrix}
    =
    \begin{bmatrix}
        u_{1,j-1}\\
        u_{2,j-1}\\
        u_{3,j-1}\\
        \vdots\\
        u_{n-3,j-1}\\
        u_{n-2,j-1}\\
        u_{n-1,j-1} + \alpha\\
    \end{bmatrix}
\]

This is a simple, linear system with a tridiagonal matrix, for which an efficient solving algorithm was developed and implemented in project 1 \cite{oblig1}.


%      _        _     _ _ _ _
%  ___| |_ __ _| |__ (_) (_) |_ _   _
% / __| __/ _` | '_ \| | | | __| | | |
% \__ \ || (_| | |_) | | | | |_| |_| |
% |___/\__\__,_|_.__/|_|_|_|\__|\__, |
%                               |___/
\subsubsection{Stability analysis}
For this scheme we have that \vref{eq:discreteRatios} for \(G_k(t_j)\) becomes
\begin{align*}
\qty(D_tG_k(t))_j &= -G_k(t_j)\mu_k \\
\frac{G_k(t_j) - G_k(t_{j-1}) }{\Delta t} &= -G_k(t_j)\mu_k \\
-G_k(t_{j-1}) &= -\Delta t G_k(t_j)\mu_k - G_k(t_j) \\
\frac{G_k(t_{j}) }{G_k(t_{j-1})} &= \frac{1}{1+\Delta t \mu_k}
\end{align*}
Since \(1+\Delta t \mu_k \geq 1\) for all \(k,\mu_k,\Delta t\) the fraction is less that one for any choice of \(\mu_k,\Delta t\). This means that Backward Euler is stable for every choice of \(\Delta x, \Delta t\) meaning we do not need, as with Forward Euler, to set restrictions on our choices of \(\Delta x \text{ and } \Delta t\).
