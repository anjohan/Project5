\section{Results in two dimensions}

\subsection{Comparison of different \(\boldsymbol{\Delta x}\) values}

\begin{figure}[H]
\input{twodim01.plt}
\caption{Simulation with \(\Delta t = 10^{-3}\) and \(\Delta x = \Delta y = 10^{-1}\). Note the difference in color scales between the figures.}
\end{figure}

\begin{figure}[H]
\input{twodim001.plt}
\caption{Simulation with \(\Delta t = 10^{-5}\) and \(\Delta x = \Delta y = 10^{-2}\).}
\end{figure}

From the figures above, it is clear that a reduction in the values of \(\Delta t\) and \(\Delta x\) reduces the error. Quantitatively, we see that the error is reduced by approximately two orders of magnitude when \(\Delta t\) is reduced from \(10^{-3}\) to \(10^{-5}\) and \(\Delta x = \Delta y\) from \(10^{-1}\) to \(10^{-2}\), which fits well with the numerical scheme having error terms proportional to \(\Delta t\), \(\qty(\Delta x)^2\) and \(\qty(\Delta y)^2\).
