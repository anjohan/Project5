\section{Implementation and code overview}
The code can be found at GitHub\footnote{\url{https://github.com/anjohan/Project5}}. See \program{makefile} for usage.

\begin{itemize}
\item \program{onedimlib.cpp} contains the class \texttt{OneDimSolver}, with methods for all the one-dimensional schemes discussed.
\item \program{onedim\_ui.cpp} is a command-line interface to the solver above, which uses its arguments to run a simulation.
\item \program{transpose.py} swaps the rows and columns in a datafile from \program{onedim\_ui.cpp} for plotting.
\item \program{tridiagonalsolver.cpp} contains the tridiagonal solver from project 1.
\item \program{deltaxtest.sh} and \program{deltaxtest.gpi} run and plot the one-dimensional simulations.
\item \program{analysis1d.py} and \program{analysis1d.gpi} run and plot the stability analysis.
\item \program{animate1d.py} takes a data file from \program{onedim\_ui.cpp} as input argument and makes an animated gif.
\item \program{simple\_diffusion.py} implements an alternative, much easier way of simulating diffusion, as discussed in \ref{sec:simplediffusion}.
\item \program{twodim.cpp} contains the explicit solver for the two-dimensional diffusion equation, and runs the simulation based on command-line arguments.
\item \program{plottwodim.gpi} plots datafiles from \program{twodim.cpp}.
\end{itemize}
To further reduce the already nearly non-existant runtime, the explicit schemes in both one and two dimensions have been parallelised using \texttt{OpenMP}.

\subsection{Visualisation}
For easy visualisation of how the concentration evolves with time, we have made the script \program{animate1d.py}, which creates an animated gif based on the data from a file generated by \program{onedim\_ui.cpp}.

See \program{README.md} (also shown at the bottom of the GitHub repository) for an example.
