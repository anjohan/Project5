%                    _       _   _           _
%   __ _ _ __   __ _| |_   _| |_(_) ___ __ _| |
%  / _` | '_ \ / _` | | | | | __| |/ __/ _` | |
% | (_| | | | | (_| | | |_| | |_| | (_| (_| | |
%  \__,_|_| |_|\__,_|_|\__, |\__|_|\___\__,_|_|
%                      |___/
\subsection{Analytical solution in two dimensions}
For the two-dimensional case, we have chosen to look at a specific set of initial and boundary conditions which make the analytical solution much easier. This analytical solution will later be used to verify that the numerical solver gives acceptably accurate results.

So, consider \vref{eq:twoDim} with the following boundaries over a spatial domain of size \(L \times L\):
\begin{equation} \label{eq:twoDimBoundaries}
\begin{aligned}
u(0,y,t) &= u(L,y,t) = 0, \quad &y \in (0,L),\quad t > 0 \\
u(x,0,t) &= u(x,L,t) = 0, \quad &x \in (0,L),\quad t > 0 \\
u(x,y,0) &= \sin(\frac{\pi}{L}x)\sin(\frac{\pi}{L}y), \quad&\text{for }x,y\in (0,L)
\end{aligned}
\end{equation}
This choice of boundary conditions has no physical interpretation, and is chosen purely because it greatly simplifies the analytical solution. These boundary conditions physically correspond to a predeposition of matter in the middle of the grid, with drains at all borders. In the steady state, it is therefore expected that all matter has left the grid, i.e. \(u(x,y)=0\) everywhere.

Solving the two-dimensional case is done in almost exactly the same way as for the one-dimensional case. Because of the similarity to what we did in \vref{sec:analyticalOneDim}, the calculation has been placed in \ref{twodimappendix}.

The analytical result is
\begin{equation} \label{eq:analyticalSolutionTwoDim}
u(x,y,t) = \sin(\frac{\pi}{L}x)\sin(\frac{\pi}{L}y)\exp{\frac{-2\pi^2}{L^2}t}
\end{equation}
