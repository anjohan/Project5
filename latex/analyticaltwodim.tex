%                    _       _   _           _
%   __ _ _ __   __ _| |_   _| |_(_) ___ __ _| |
%  / _` | '_ \ / _` | | | | | __| |/ __/ _` | |
% | (_| | | | | (_| | | |_| | |_| | (_| (_| | |
%  \__,_|_| |_|\__,_|_|\__, |\__|_|\___\__,_|_|
%                      |___/
\subsection{Analytical solution of the diffusion equation}
For the two dimensional case we will not consider a general solution to \vref{eq:twoDim}. However, we will consider a specific problem of our choice such that we can compare our two dimensional solver with the solution, ensuring us that the solver gives us sufficient accurate results.

So, consider \vref{eq:twoDim} with the following boundaries over a spatial domain of size \(L \times L\):
\begin{equation} \label{eq:twoDimBoundaries}
\begin{aligned}
u(0,y,t) &= u(L,y,t) = 0, \quad &y \in (0,L),\quad t > 0 \\
u(x,0,t) &= u(x,L,t) = 0, \quad &x \in (0,L),\quad t > 0 \\
u(x,y,0) &= \sin(\frac{\pi}{L}x)\sin(\frac{\pi}{L}y), \quad&\text{for }x,y\in (0,L)
\end{aligned}
\end{equation}
This choice of boundary conditions has no physical interpretation, and is chosen purely because it greatly simplifies the analytical solution.

The procedure is similar as for the one dimensional case. Since the procedure of finding a solution is very similar as what we did in \vref{sec:analyticalOneDim}, the calculation has been placed in \ref{twodimappendix}.

The analytical solution of \vref{eq:twoDim} with the boundaries \vref{eq:twoDimBoundaries} derived in \ref{twodimappendix} is
\begin{equation} \label{eq:analyticalSolutionTwoDim}
u(x,y,t) = \sin(\frac{\pi}{L}x)\sin(\frac{\pi}{L}y)\exp{\frac{-2\pi^2}{L^2}t}
\end{equation}
