%                                   _ _
%   __ _ _ __  _ __   ___ _ __   __| (_)_  __
%  / _` | '_ \| '_ \ / _ \ '_ \ / _` | \ \/ /
% | (_| | |_) | |_) |  __/ | | | (_| | |>  <
%  \__,_| .__/| .__/ \___|_| |_|\__,_|_/_/\_\
%       |_|   |_|

\section{Solving for the two dimensional case}\label{twodimappendix}
Here, we will explain how we found the solution \vref{eq:analyticalSolutionTwoDim} with the given problem \vref{eq:twoDim} with the boundaries \vref{eq:twoDimBoundaries}.


We make an ansatz that
\[
u(x,y,t) = F(x,y)G(t)
\]
\Vref{eq:twoDim} gives us that
\begin{align}
G(t)\qty(\pdv[2]{F(x,y)}{x}+ \pdv[2]{F(x,y)}{y} ) &= F(x,y)\pdv{G(t)}{t} \nonumber\\
\frac{1}{F(x,y)}\qty(\pdv[2]{F(x,y)}{x}+ \pdv[2]{F(x,y)}{y} ) &= \frac{G'(t)}{G(t)} \label{eq:solveFrom}
\end{align}
Since the ratios of two functions with different variables are equal, we must have that they are all equal to a constant \(-\lambda\). We will first solve for \(F(x,y)\).

Assume that we can again separate \(F(x,y)\) into
\[
F(x,y) = X(x)Y(y)
\]
Inserting this into \vref{eq:solveFrom} gives us
\begin{align*}
\frac{Y(y)X''(x)+ Y''(y)X(x)}{X(x)Y(y)} &= -\lambda \\
\frac{X''}{X}+ \frac{Y''}{Y} &= -\lambda
\end{align*}
Since an addition of the ratios of two functions with different variables gives a constant, each of the ratios must therefore be a constant.We will first solve for \(X\):
\begin{align}
\frac{X''}{X} &= -\qty(\frac{Y''}{Y} +\lambda) \label{eq:mysteriousMu}\\
\frac{X''}{X} &= -\mu \nonumber
\end{align}
This gives a general solution
\[
X(x) = c_1\cos(\sqrt{\mu}x) + c_2\sin(\sqrt{\mu}x)
\]
Since \(u(0,y,t) = u(L,y,t) = 0\) from \vref{eq:twoDimBoundaries}, we must have that \(c_1 = 0\) and
\begin{align}
\sqrt{\mu_k}L &= k\pi \nonumber , \quad k \in \mathbb{N}\\
\mu_k L^2 &= \qty(k\pi)^2 \nonumber \\
 \mu_k  &=\qty(\frac{k\pi}{L})^2 \label{eq:twoDimMu}
\end{align}
which gives that \(c_2 = c_k\), and
\[
X(x) = c_k\sin(\frac{k\pi}{L}x)
\]
Now, we have to solve for \(Y\):

From \vref{eq:mysteriousMu} and \vref{eq:twoDimMu}, we have that
\begin{align*}
\frac{Y''}{Y} +\lambda  &=\qty(\frac{k\pi}{L})^2 \\
\frac{Y''}{Y}&=-\qty(-\qty(\frac{k\pi}{L})^2 + \lambda ) \\
\frac{Y''}{Y}&=-\nu
\end{align*}
This gives a general solution of \(Y\):
\[
Y(y) = d_1\cos(\sqrt{\nu}y) + d_2\sin(\sqrt{\nu}y)
\]
By using the same procedure of finding \(X\), we get that \(d_1 = 0\) and
\[
\nu = \qty(\frac{m\pi}{L})^2
\]
for \(m \in \mathbb{N}\). Setting \(d_2 = d_m\), we get the solution of \(Y\) to be
\[
Y(y) = d_m\sin(\frac{m\pi}{L}y)
\]
with
\begin{equation}\label{eq:notSoMysteriousNu}
\nu_m = \qty(\frac{m\pi}{L})^2
\end{equation}
Therefore, we have the particular solution of F for any value of \(k,m\).
\[
F_{k,m}(x,y) = c_kd_m\sin(\frac{k\pi}{L}x)\sin(\frac{m\pi}{L}y)
\]

In \vref{eq:solveFrom}, we can solve for \(G(t)\):
\[
\frac{G'(t)}{G(t)}  = -\lambda
\]
which gives the general solution
\[
G(t) = Ce^{-\lambda t}
\]
where we set the constant \(C = 1\). To find \(\lambda\) we just the expression of \(\nu_m\) which we found in \vref{eq:notSoMysteriousNu}:
\begin{align*}
\nu_m &= \qty(\frac{m\pi}{L})^2 \\
-\qty(\frac{k\pi}{L})^2 + \lambda_{k,m} &= \qty(\frac{m\pi}{L})^2  \\
\lambda_{k,m} &= \qty(\frac{m\pi}{L})^2+\qty(\frac{k\pi}{L})^2
\end{align*}
we see that the value of \(\lambda \) depends on the value of \(k\) and \(m\), thus giving that \(\lambda = \lambda_{k,m}\).

In total, we have the following particular solution of \(u\):
\[
u_{k,m}(x,y,t) =  c_kd_m\sin(\frac{k\pi}{L}x)\sin(\frac{m\pi}{L}y)e^{-\lambda_{k,m} t}
\]
To decide \(c_k\) and \(d_m\), we have to look at the boundary of \(u\) at \(t = 0\), which gives
\[
\sin(\frac{\pi}{L}x)\sin(\frac{\pi}{L}y) =  c_kd_m\sin(\frac{k\pi}{L}x)\sin(\frac{m\pi}{L}y)
\]
meaning that \(c_k = d_m = k = m = 1\) and therefore
\[
\lambda_{k,m} = 2\qty(\frac{\pi}{L})^2
\]

This gives the final solution of our problem described in \vref{eq:twoDim} and \vref{eq:twoDimBoundaries} to be
\[
u(x,y,t) = \sin(\frac{\pi}{L}x)\sin(\frac{\pi}{L}y)e^{-2\qty(\frac{\pi}{L})^2 t}
\]
