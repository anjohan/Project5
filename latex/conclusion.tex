\section{Conclusion}
We have now seen how the diffusion equation can be solved numerically in one and two dimensions. In both situations, it is clear that an explicit algorithm is easy to derive and implement and will give reasonable results when the spatial and temporal step lengths are chosen wisely. On the other hand, it has also become apparent that an unwise choice can lead to spectacularly wrong results, while the implicit methods discussed give stable results no matter what step lengths are chosen, with the Crank-Nicolson scheme being the most accurate.

Since we were able to reuse an efficient tridiagonal solver from a previous project, the added difficulties in derivation and implementation of the implicit methods are clearly worth the extra effort when simulating diffusion in a finer mesh of spatial points for which the criteria of the explicit scheme would require a lot of CPU time.

As such, future work should include the development of an implicit scheme for the two-dimensional diffusion equation. Additionally, it would be beneficial to either analytically derive or numerically search for the stability requirement of the explicit scheme as done for one-dimensional equation. For practical applications it might also be beneficial to determine analytical solutions for initial and boundary conditions corresponding to other physical situations.

Considering that the 25 or so pages and about 400 lines of code we have written about one-dimensional diffusion could be replaced by the two paragraphs and 17 lines of Python code in \ref{sec:simplediffusion}, it might also be wise to look at other ways of studying diffusion, for example Monte Carlo simulations.
