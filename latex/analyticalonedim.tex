\subsection{Analytical solution of the diffusion equation}\label{sec:analyticalOneDim}
Consider the diffusion equation with the given boundaries:
\begin{equation}\label{eq:problem}
\begin{aligned}
\pdv[2]{u(x,t)}{x} &= \pdv{u(x,t)}{t} &\: t>0, \:\:x\in[0,L] \\
u(x,0) &= g(x)  &0 < x < L\\
u(0,t) &=  a & a \in \mathbb{R},  t>0 \\
u(L,t) &= b &b \in \mathbb{R}, t>0
\end{aligned}
\end{equation}
As will become clear later, life is much easier when \(a\) and \(b\) are both zero. As such, we introduce a new function
\[
v(x,t) = u(x,t) + w(x,t)
\]
where \(u\) is a solution of \ref{eq:problem} and the following also holds:
\begin{equation}\label{eq:dummy}
    \begin{aligned}
        \pdv[2]{v(x,t)}{x} &= \pdv{v(x,t)}{t} &\: t>0, \:\:x\in[0,L] \\
        v(x,0) &= f(x)  &0 < x < L\\
        v(0,t) &=  0 &  t>0 \\
        v(L,t) &= 0 & t>0
    \end{aligned}
\end{equation}
When introducing \(v(x,t)\), a new problem arises: What does \(w(x,t)\) actually look like? Inserting the definition of \(v\) into \ref{eq:dummy} gives
\begin{alignat*}{2}
    \pdv[2]{v(x,t)}{x} &= \pdv{v(x,t)}{t}  \\
    \pdv[2]{u(x,t)}{x} + \pdv[2]{w(x,t)}{x} &= \pdv{u(x,t)}{t} + \pdv{w(x,t)}{t}
\end{alignat*}
The assumption that \(u\) is a solution to the original problem, which means that
\[
    \pdv[2]{u(x,t)}{x} = \pdv{u(x,t)}{t}
\]
As a result, \ref{eq:dummy} will hold if
\begin{equation} \label{eq:dummy1}
    \pdv[2]{w(x,t)}{x} = 0 = \pdv{w(x,t)}{t}
\end{equation}
These two equalities are separately fulfilled if
\begin{alignat*}{2}
	\pdv[2]{w(x,t)}{x}  &= 0\\
	\pdv{w(x,t)}{x} &= A + h_1(t)\\
	w(x,t) &= Ax +xh_1(t) + B + h_2(t) \eqtag{eq:wX}
\shortintertext{and}
    \pdv{w(x,t)}{t} &= 0\\
    w(x,t) &= C + h_3(x)\eqtag{eq:wT}
\intertext{Combining these two requirements for \(w\), we see that it must be on the form}
    w(x,t) &= w(x) = Ax + D
\end{alignat*}
because there is no dependence of \(t\) in \vref{eq:wT}.

We have now found a possible form for \(w\). However, the values of \(A\) and \(D\) still need to be determined. This can be done by looking at the boundary conditions for \(v\) at \(x = 0 \) and \(x = L\) and using the assumption that \(u\) fulfills the boundary conditions of \ref{eq:problem}, i.e. \(u(0,t)=a\):
\[
    0 = v(0,t) = u(0,t) + w(x=0) = a+D
\]
which implies \(D = -a\). Similarly, the boundary at \(x=L\) gives
\[
    0 = v(L,t) = u(L,t) + w(x=L) = b+AL-a \implies A = \frac{a-b}{L}
\]
In conclusion, we have
\[
    w(x) = \frac{a-b}{L}x - a
\]
As the expression of \(w\) has been found, we have also found \(f(x)\) in \vref{eq:dummy}:
\begin{align*}
	v(x,0) &= u(x,0) + w(x) \\
	&= g(x)+w(x) \\
	&=f(x)
\end{align*}
Having set up the problem properly, we are now ready to solve for \(v(x,t)\). To do so, we make the assumption that \(v\) is separable, i.e. it can be written as
\[
    v(x,t) = F(x)G(t)
\]
which means that
\begin{alignat*}{2}
    \pdv[2]{v(x,t)}{x} &= \pdv{v(x,t)}{t}\\
    F''(x)G(t) &= F(x)G'(t)\\
    \frac{F''(x)}{F(x)} &= \frac{G'(t)}{G(t)}
\end{alignat*}
As the two sides of the last equation depend on different variables which can be varied independently, both sides must equal some constant \(-\lambda\):
\[
    \frac{F''(x)}{F(x)} = \frac{G'(t)}{G(t)} = -\lambda\eqtag{eq:ration}
\]
Solving for \(F\) we get
\[
	F'' = -\lambda F
\]
with a general solution
\[
    F(x) = c_1\cos(x\sqrt{\lambda} ) + c_2\sin(x\sqrt{\lambda})
\]
To find the constants \(c_1\) and \(c_2\), we will use the boundaries for \(v(x,t)\) at \(x = 0\) and \(x = L\). The advantage of having the boundary conditions equal to zero now becomes apparent, as this implies
\begin{alignat*}{8}
    0 &= v(0,t) &= F(0)G(t) \implies F(0) &= 0
    \shortintertext{and}
    0 &= v(L,t) &= F(L)G(t) \implies F(L) &= 0
\end{alignat*}
Since \(\sin(0)=0\neq\cos(0)\), we must have that
\begin{alignat*}{2}
    0 &= c_1
    \shortintertext{and}
    0 &=  c_2\sin(L\sqrt{\lambda})
\intertext{Of course, \(c_2\) could also be set to \(0\), but this is the trivial solution which is of no interest. Therefore, we must have}
    0 &= \sin(L\sqrt{\lambda})
\end{alignat*}
Since sine is zero for every integer multiple of \(\pi\), we have that
\begin{alignat*}{2}
    L\sqrt{\lambda} &= k\pi \quad k\in \mathbb{N}\\
    \lambda &= \qty(\frac{k \pi}{L})^2
\end{alignat*}
In conclusion, the following is a solution for all \(k\in\mathbb{N}\):
\[
    F_k(x) = c_k\sin(x\frac{k \pi}{L})
\]
Similarly, \ref{eq:ration} implies that
\[
    G_k'(t) = -\lambda_k G_k(t) \implies
    G_k(t) = C_k\exp{-\qty(\frac{k \pi}{L})^2t}
\]
For simplicity, \(C_k\) can be absorbed into \(c_k\).

We are now close to an expression for \(v\):
\[
    v_k(x,t) = c_k\exp{-\qty(\frac{k \pi}{L})^2t}\sin(x\frac{k \pi}{L})
\]
What remains is to find \(c_k\). The final, unused piece of information is the initial condition, i.e. \(v(x,0)\). Inserting \(t=0\) in the equation above makes the exponential \(1\), i.e.
\[
    f(x) = c_k\sin(x\frac{k \pi}{L})
\]
Multiplying  the equation above by \(\sin(x\frac{k \pi}{L})\), integrating both sides and using the orthogonality of sine (note that this sine is \(L\)-periodic), gives us
\begin{align*}
    \int_0^Lf(x)\sin(x\frac{k \pi}{L})\dd{x} &= c_k\frac{L}{2} \\
    \frac{2}{L}\int_0^Lf(x)\sin(x\frac{k \pi}{L})\dd{x} &= c_k
\end{align*}
However, we are not looking for one particular solution of \ref{eq:dummy}, but for all solutions. From the principle of superposition, every particular solution contributes to another solution. Specifically, this means that
\[
    v(x,t) = \sum_{k=1}^\infty v_k(x,t)
\]

Remember now the original ansatz
\[
    v(x,t) = u(x,t) + w(x,t)
\]
As \(v\) and \(w\) have now been determined, \(u\) can be found through a reorganisation:
\begin{align*}
    u(x,t) &= v(x,t)- w(x,t) \\
    &=  \sum_{k=1}^\infty v_k(x,t) - w(x) \\
    &=\sum_{k=1}^\infty c_k\exp{-\qty(\frac{k \pi}{L})^2t}\sin(x\frac{k \pi}{L})  -\qty(\frac{a-b}{L}x - a)
\end{align*}
with
\[
    c_k = \frac{2}{L}\int_0^Lf(x)\sin(x\frac{k \pi}{L})\dd{x}
\]
\hfill \\
For our case, that is \vref{eq:diffusion}, we have that \(a = 0\), \(b = 1\), \(g(x) = 0\) and \(L = 1\). Inserting this into the solution of \vref{eq:dummy} gives us
\begin{equation}\label{eq:analyticalSolution}
	u(x,t) = \sum_{k=1}^\infty c_k\exp{-(k\pi)^2t}\sin(xk\pi)  + x
\end{equation}
with
\begin{align*}
    c_k &= 2\int_0^1(0-x)\cdot\sin(xk\pi)\dd{x}
    = - \frac{2}{k\pi}\qty[\frac{\sin(k\pi x)}{k\pi} - x\cos(k\pi x)]_0^1
    = \frac{2(-1)^{k+2}}{k\pi}
    =  \frac{2(-1)^k}{k\pi}
\end{align*}
