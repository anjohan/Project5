%  _ ____                            _ _        _
% / |  _ \       _ __ ___  ___ _   _| | |_ __ _| |_ ___ _ __
% | | | | |_____| '__/ _ \/ __| | | | | __/ _` | __/ _ \ '__|
% | | |_| |_____| | |  __/\__ \ |_| | | || (_| | ||  __/ |
% |_|____/      |_|  \___||___/\__,_|_|\__\__,_|\__\___|_|
%
\section{Results in one dimension}
\subsection{Qualitative overview for different \(\boldsymbol{\Delta x}\)}
As derived in \ref{forwardstability}, the Forward Euler scheme requires that \(\Delta t \leq \tfrac{1}{2}\qty(\Delta x)^2\). Simulations with \(\Delta x = 1/10\) and \(\Delta x = 1/100\) and \(\Delta t = \tfrac{1}{2}\qty(\Delta x)^2\) gives the figure below.
\begin{figure}[H]
\centering
\input{deltaxtest.plt}
\caption{Results for two different values of \(\Delta x\). Generated by \program{deltaxtest.sh}.}
\end{figure}
It is clear from the figure that all methods give acceptable results for the tested values of \(\Delta x\). As expected from the error terms, the Crank Nicolson scheme slightly outperforms the others. For \(\Delta x=1/100\), there is essentially no difference. We see that the steady state requirement of linearity from \ref{steadystate} is fulfilled.

\subsection{Quantitative error analysis}
\begin{figure}[H]
\centering
\input{error1d.plt}
\caption{A closer look at the errors in the previous figure.}
\end{figure}
From the derivation of the methods and their error terms, we know that the Euler schemes have one error term proportional to \(\Delta t\) and one proportional to \(\Delta x^2\), while the Crank-Nicolson scheme has error terms proportional to \(\Delta t^2\) and \(\Delta x^2\). Since \(\Delta t\) has been set to \(\tfrac{1}{2}\Delta x^2\), the error in the Euler schemes should decrease by two orders of magnitude when \(\Delta x\) is reduced from \(\num{0.1}\) to \(\num{0.01}\), which fits well by the computed errors. As expected, the Crank-Nicolson error decreases more rapidly, but an exact prediction can not be made since the error term proportional to \(\Delta x^2\) should decrease by two orders of magnitude and the term proportional to \(\Delta t^2\) by four.


\subsection{Stability analysis}
\begin{figure}[H]
\centering
\input{stability1d.plt}
\caption{Visual verification of the criterion for the ratio \(\Delta t\) and \(\Delta x\) found through von Neumann analysis, with \(\Delta x = \num{0.1}\). The analytically derived \(\Delta t/\qty(\Delta x)^2\leq1/2\) fits well with the result for the Forward Euler scheme, while the other schemes do not rely on this ratio, as expected. Generated by \program{analysis1d.py}.}
\end{figure}
