%  _ ____                            _ _        _
% / |  _ \       _ __ ___  ___ _   _| | |_ __ _| |_ ___ _ __
% | | | | |_____| '__/ _ \/ __| | | | | __/ _` | __/ _ \ '__|
% | | |_| |_____| | |  __/\__ \ |_| | | || (_| | ||  __/ |
% |_|____/      |_|  \___||___/\__,_|_|\__\__,_|\__\___|_|
%
\section{Results in one dimension}
\subsection{Qualitative overview for different \(\boldsymbol{\Delta x}\)}
As derived in \ref{forwardstability}, the Forward Euler scheme requires that \(\Delta t \leq \tfrac{1}{2}\qty(\Delta x)^2\). Simulations with \(\Delta x = 1/10\) and \(\Delta x = 1/100\) and \(\Delta t = \tfrac{1}{2}\qty(\Delta x)^2\) gives the figure below.
\begin{figure}[H]
\centering
\input{deltaxtest.plt}
\caption{Results for two different values of \(\Delta x\). Generated by \program{deltaxtest.sh}.}
\end{figure}
It is clear from the figure that all methods give acceptable results for the tested values of \(\Delta x\). As expected from the error terms, the Crank Nicolson scheme slightly outperforms the others. For \(\Delta x=1/100\), there is essentially no difference. We see that the steady state requirement of linearity from \ref{steadystate} is fulfilled.

\subsection{Quantitative error analysis}
\begin{figure}[H]
\input{error1d.plt}
\caption{Maximum error as a function of \(\Delta t\) with \(\Delta x=\num{0.1}\), for values below the stability limit.}
\end{figure}


\subsection{Stability analysis}
\begin{figure}[H]
\centering
\input{stability1d.plt}
\caption{Visual verification of the criterion for the ratio \(\Delta t\) and \(\Delta x\) found through von Neumann analysis, with \(\Delta x = \num{0.1}\). The analytically derived \(\Delta t/\qty(\Delta x)^2\leq1/2\) fits well with the result for the Forward Euler scheme, while the other schemes do not rely on this ratio, as expected. Generated by \program{analysis1d.py}.}
\end{figure}
